Implements factor elimination in polynomials based on factorization and variable bounds.

\begin{description}
	\item[Input] Arbitrary formulas.
	\item[Output] Arbitrary formulas.
	\item[Lemmas] None.
\end{description}

Given a constraint $p \sim 0$ with $\sim \in \{ =, \neq, \geq, >, \leq, < \}$ and bounds $b$ on the variables in $p$, the module does the following:
It computes a factor $q$ such that $p = q \cdot r$ and evaluates $q$ on the intervals represented by the bounds.
If the resulting interval $q(b)$ if sign-invariant, the constraint can be simplified or additional constraints can be added.
We consider an interval to be sign-invariant, if it is positive, semi-positive, zero, semi-negative or negative.

\begin{table}
\begin{center}
\caption{Simplifications for $q \cdot r \sim 0$}
\begin{tabular}{|c||c|c|c|c|c|}
\hline
$\sim$ & $q(b) > 0$ & $q(b) \geq 0$ & $q(b) = 0$ & $q(b) < 0$ & $q(b) \leq 0$ \\
\hline\hline
$=$ & \multirow{6}{*}{$p := r$} & $f := q=0 \vee r=0$ & \multirow{6}{*}{$p := 0$} & \multirow{2}{*}{$p := r$} & $f := q=0 \vee r=0$ \\
\cline{1-1}\cline{3-3}\cline{6-6}
$\neq$ & & $f := q>0 \wedge r \neq 0$ & & & $f := q<0 \wedge r \neq 0$ \\
\cline{1-1}\cline{3-3}\cline{5-6}
$\geq$ & & $f := q=0 \vee r \geq 0$ & & $c := r \leq 0$ & $f := q=0 \vee r \leq 0$ \\
\cline{1-1}\cline{3-3}\cline{5-6}
$>$ & & $f := q>0 \wedge r>0$ & & $c := r<0$ & $f := q<0 \wedge r<0$ \\
\cline{1-1}\cline{3-3}\cline{5-6}
$\leq$ & & $f := q=0 \vee r \leq 0$ & & $c := r \geq 0$ & $f := q=0 \vee r \geq 0$ \\
\cline{1-1}\cline{3-3}\cline{5-6}
$<$ & & $f := q>0 \wedge r<0$ & & $c := r>0$ &  $f := q<0 \wedge r>0$ \\
\hline
\end{tabular}
\caption{Notation: $p := p'$ replaces the polynomial, $c := p' \sim 0$ replaces the whole constraint by a new one and $f := f'$ replaces the constraint by a new formula.}
\end{center}
\end{table}

To maximize the cases where $q(b)$ actually is sign-invariant, we proceed as follows.
We compute a factorization of $p$, that is a number of polynomials $p_i$ such that $p = \prod_{i=1}^k p_i$.
We now separate all $p_i$ into two sets: $P_q$ for all sign-invariant $p_i$ and $P_r$ for all other $p_i$.
Thereby, we set $q = \prod_{t \in P_q} t$ and $r = \prod_{t \in P_r}$ and know that $q(b)$ is sign-invariant.

Instead of computing $q(b)$ once again afterwards, we can determine the type of sign-invariance of $q(b)$ from the types of sign-invariances of the factors from $P_q$.
Assume that we start with a canonical factor $1$ and a sign-invariance of $>$, we can iteratively combine them like this:

\begin{table}
\begin{center}
\caption{Combining types of sign-invariance}
\begin{tabular}{|c||c|c|c|c|c|}
\hline
$\cdot$ & $>$ & $\geq$ & $=$ & $\leq$ & $<$ \\
\hline\hline
$>$ & $>$ & $\geq$ & $=$ & $\leq$ & $<$ \\
\hline
$\geq$ & $\geq$ & $\geq$ & $=$ & $\leq$ & $\leq$ \\
\hline
$=$ & $=$ & $=$ & $=$ & $=$ & $=$ \\
\hline
$\leq$ & $\leq$ & $\leq$ & $=$ & $\geq$ & $\geq$ \\
\hline
$<$ & $<$ & $\leq$ & $=$ & $\geq$ & $>$ \\
\hline
\end{tabular}
\end{center}
\end{table}
