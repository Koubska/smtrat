\chapter{Implementing further modules}
\label{chapter:implementingamodule}
In this chapter we explain how to implement further modules. One
principle of the design of \smtrat is that you can take advantage of
all the features it provides, but you do not have to.

\section{Quickstart}
Figure~\ref{fig:rudiimplementation} shows the most rudimentary way of how a module
can be implemented. Furthermore, there are a few more things which must be added to
\smtrat:
\begin{enumerate}
	\item Add the module type \texttt{MT\_MyModule} to the \texttt{enum}
		\texttt{ModuleType} of all module types in \texttt{src/ModuleType.h}.
	\item The file \texttt{src/modules/Modules.h} has to include the header of the new module, i.e.
		add ``\texttt{\#include "MyModule.h"}'' to  \texttt{modules/Modules.h}.
	\item Extend the mapping of module types to module factories in the constructor
		of the class \managerClass:
\begin{verbatim}
addModuleType
(
    MT_MyModule,
    new StandardModuleFactory<MyModule>()
);
\end{verbatim}
	\item Add a proposition to the class \texttt{src/Formula.h} which expresses for a
		formula	that your module cannot solve it, that is returns
		\texttt{unknown} or calls its backend
		(for further details to this see Section~\ref{sec:runbackend}). Each existing
		proposition is represented by a bit in a \texttt{bitset} within \texttt{Formula},
		currently of size $64$.	Hence, the proposition to add has to take an unused bit.
		If all bits are used the bitset must be extended.
		An example of how to add this proposition is given by
		Figure~\ref{fig:proposition}.
	\item Extend the method \texttt{Formula::notSolvableBy}
		by the case that \texttt{\_moduleType} is \texttt{MT\_MyModule} returning
		the before created proposition as shown in Figure~\ref{fig:notSolvableBy}
	\item Extend the method \texttt{Formula::resetSolvableByModuleFlags}\ldots

	\item Finally, the file \texttt{src/CMakeLists.txt} has to be updated. Add to the set
		of \texttt{lib\_modules\_headers} ``\texttt{src/modules/MyModule.h}'' and to the set
		of \texttt{lib\_modules\_src} ``\texttt{src/modules/MyModule.cpp}''.
\end{enumerate}

\begin{figure}[htb]
\caption{The most rudimentary way of a module can be implemented.}
\label{fig:rudiimplementation}
\begin{verbatim}
#include "../Module.h"

class MyModule : public Module
{
public:
    MyModule( Manager* const _manager )
        : Module( _Manager )
    { mModuleType	= MT_MyModule; }

    ~MyModule( ) { }
};
\end{verbatim}
\end{figure}

\begin{figure}[htb]
\caption{Example of how to add a proposition to \texttt{Formula.h}.}
\label{fig:proposition}
\begin{verbatim}

..
    static const Condition PROP_CANNOT_BE_SOLVED_BY_MYMODULE
        = Condition().set( 53, 1 );
..
\end{verbatim}
\end{figure}

\begin{figure}[htb]
\caption{How to extend the method \texttt{Formula::notSolvableBy} in \texttt{Formula.cpp}.}
\label{fig:notSolvableBy}
\begin{verbatim}
void Formula::notSolvableBy( ModuleType _moduleType )
{
    switch( _moduleType )
    {
	    ..
	    case MT_MyModule:
	    {
	        mPropositions |= PROP_CANNOT_BE_SOLVED_BY_MYMODULE;
	        break;
	    }
	    ..
    }
}
\end{verbatim}
\end{figure}

The generated module now uses the straightforward implementation of the superclass
\texttt{Module}. This includes many useful methods to read and
manipulate the data in the module and the interfaces which can be extended
such that they contain the actual implementation of your module.

\section{Members of a modul}
Before we explain the virtual interfaces containing the actual
implementation of your modul we give a small description of the members
of \texttt{Module}.

\begin{itemize}
	\item \begin{verbatim}vector< unsigned > mBackTrackPoints\end{verbatim}
		Stores the backtrack points.
	\item \begin{verbatim}vector< Formula > mInfeasibleSubsets\end{verbatim}
		Stores the infeasible subsets of the so far received constraints, if the module determined their
		inconsistency before. An object of the type \texttt{Formula} currently consists
		of a conjunction of constraints and stores which of the propositions, defined in
		Chapter~\ref{chapter:composingats}, hold.
	\item \begin{verbatim}Manager* const mpManager\end{verbatim}
		A pointer to the manager which stores this module. By this pointer the modul can obtain
		its backend, which gets assigned by the manager as it implements the strategy.
	\item \begin{verbatim}ModuleType mModuleType\end{verbatim}
		The type of this module.
	\item \begin{verbatim}vector< Module* > mBackend\end{verbatim}
		The pointer to the module's backend. The backend is stored in the manager.
	\item \begin{verbatim}vector< unsigned > mBackendBacktrackpoints\end{verbatim}
		The backtrack points in the backend.
	\item \begin{verbatim}bool mBackendUptodate\end{verbatim}
		A flag which indicates that the current backend considers the so
		far passed constraints.
	\item \begin{verbatim}Formula* mpReceivedConstraints\end{verbatim}
		The vector of the so far received constraints.
	\item \begin{verbatim}Formula* mpPassedConstraints\end{verbatim}
		The vector of the so far passed constraints.
\end{itemize}

\section{Virtual interfaces}
In the following we explain the virtual interfaces
of the module, showing what their implementation in \texttt{Module} already
provides and how they should be extended in your module.

\subsection{Informing about a constraint}
\begin{verbatim}void inform( Constraint* _constraint )\end{verbatim}
Informs the module about the existance of the given constraint. This information might
be useful for the module. The implementation in the \texttt{Module} does just nothing by calling
this method.

\begin{figure}[htb]
\label{fig:exa_inform}
\caption{Example showing how to implement the method \texttt{inform}}
\begin{verbatim}
	void MyModule::inform( Constraint* _constraint )
	{
	    /*
	     * Write the implementation here.
	     */
	}
\end{verbatim}
\end{figure}

\subsection{Adding a constraint}
\begin{verbatim}bool addConstraint( Constraint* _constraint )\end{verbatim}
Adds a constraint to the vector of the so far received constraints of the module. If the module
can already decide whether the given constraint is not consistent, it returns \false
otherwise \true. This is exactly how it is implemented in \texttt{Module}, unless
it does no kind of consistency check on the given constraint, that is it always adds the
given constraint to the vector of the so far received constraints and returns \true.
Note, that implementing your own module might need some initialization in this method,
but you should always call \texttt{Module::addConstraint} at the beginning of your
implementation of \texttt{addConstraint}.

\begin{figure}[htb]
\label{fig:exa_addconstraint}
\caption{Example showing how to implement the method \texttt{addConstraint}}
\begin{verbatim}
	bool MyModule::addConstraint( Constraint* _constraint )
	{
	    Module::addConstraint( _constraint );
	    /*
	     * Write the implementation here.
	     */
	}
\end{verbatim}
\end{figure}

\subsection{Checking for consistency}
\begin{verbatim}Answer isConsistent()\end{verbatim}
Implements the actual consistency check of the constraints this module has so far received.
There are three options how this module can answer: it can determine that these constraints
are consistent and returns \true, or it determines inconsistency and returns
\false. Otherwise, it returns \texttt{unknown}. If this module has a backend, it can call it
to check the consistency of any set of constraints. How to run the backend is shown in
Section~\ref{sec:runbackend}. The method \texttt{Module::isConsistent} does nothing more than directly
calling its backend on the
so far received constraints. A module has also the opportunity to reason about the conflicts
occured, if it determines inconsistency. For this purpose it can store several infeasible
subsets of the set of so far received constraints, which can be accessed as seen before.

\begin{figure}[htb]
\label{fig:exa_isconsistent}
\caption{Example showing how to implement the method \texttt{isConsistent}}
\begin{verbatim}
	Answer MyModule::isConsistent( )
	{
	    /*
	     * Write the implementation here.
	     */
	}
\end{verbatim}
\end{figure}

\subsection{Pushing a backtrack point}
\begin{verbatim}void pushBacktrackPoint()\end{verbatim}
Adds a backtrack point to the stack of backtrackpoints. This is already implemented by
\texttt{Module::pushBacktrackPoint}. It might be necessary to store the current state
of the datastructure used in your module.

\begin{figure}[htb]
\label{fig:exa_pushBacktrackPoint}
\caption{Example showing how to implement the method \texttt{pushBacktrackPoint}}
\begin{verbatim}
	void MyModule::pushBacktrackPoint( )
	{
	    Module::pushBacktrackPoint();
	    /*
	     * Write the implementation here.
	     */
	}
\end{verbatim}
\end{figure}

\subsection{Popping a backtrack point}
\begin{verbatim}void popBacktrackPoint()\end{verbatim}
Removes the last backtrack point from the stack of backtrack points and undoes everything
which has been done after adding that backtrack point. \newline\texttt{Module::popBacktrackPoint}
already removes the corresponding constraints from the vector of the so far received
constraints, applies \texttt{popBacktrackPoint} in its backend until it does not
contain any constraint created as a consequence of a removed received constraint, and
removes all infeasible subsets containing removed received constraints. If you implement
\texttt{popBacktrackPoint} make sure that everything stored additionally, which
depends on the removed received constraints, gets deleted. Call
\texttt{Module::popBacktrackPoint} afterwards.

\begin{figure}[htb]
\label{fig:exa_popBacktrackPoint}
\caption{Example showing how to implement the method \texttt{popBacktrackPoint}}
\begin{verbatim}
	void MyModule::popBacktrackPoint( )
	{
	    /*
	     * Write the implementation here.
	     */
	    Module::popBacktrackPoint();
	}
\end{verbatim}
\end{figure}

\section{Running the backend module}
\label{sec:runbackend}
Fortunately, there is no need to manage the addition of constraints to the backend, or pushing and
popping its backtrack points. This would be even more involved as we do allow changing the
backend if it is appropiate (more details to this are explained in Chapter~\ref{chapter:composingats}).
Running the backend is done in two steps:
\begin{enumerate}
	\item Fill up the vector of the so far passed constraints.
	\item Call \texttt{runBackend()}.
\end{enumerate}
The first step is a bit more tricky, as we need to know which received constraints led to a passed
constraint. For this purpose a modul contains a mapping from a passed constraint to one or more
sets of received constraints. We give a small example by showing the behavior of one of the modules
we provide, the \simplifierModuleClass. Let us assume that this modul has so far received the following
constraints:
$$c_0:x\leq0,\ c_1:x\geq 0,\ c_2:x=0$$
\simplifierModuleClass combines the first two constraints $c_0$ and $c_1$ to $c_3:x=0$. Then it combines
$c_3$ with $c_2$ to $c_5:x=0$. Afterwards it calls its backend on the only remaining constraint,
that is the passed constraints contain only $c_5:x=0$. The mapping of $c_5$ to the received constraints it
stems from is $$c_5\ \mapsto\ (\{c_0,\ c_1\},\ \{c_2\}).$$

The mapping is maintained automatically and offers two methods to add constraints to the vector
of the passed constraints:
\begin{itemize}
	\item \begin{verbatim}bool addReceivedConstraintToPassedConstraints( unsigned _pos )\end{verbatim}
		Adds the constraint at the given positition of the vector of the so far received constraints
		to the vector of the passed constraints. The mapping to its \emph{original constraints} contains
		only the constraint at the given position in the vector of received constraints. It returns
		\true if a constraint has been added, which only occurs if the given position is
		indeed within the range of the vector of received constraints and this constraint has not
		yet been added.
	\item
		\begin{verbatim}
		bool addConstraintToPassedConstraints
		(
		    Constraint*                  _constraint,
		    vector< set< Constraint* > > _origins
		)
		\end{verbatim}
		Adds the given constraint to the vector of the passed constraints. It is mapped to the given
		original constraints \texttt{\_origins}. It returns \true if a constraint has been
		added, which only occurs if the given constraint has not yet been added. Note, that
		\texttt{\_origins} must contain only members of the vector of so far received constraints. We
		do not check this by reason of efficiency.
\end{itemize}

