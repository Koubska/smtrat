\section{Constructing an formula}
\label{chapter:constructingaformula}
The class \formulaClass represents SMT formulas, which are
defined according to the following abstract grammar

\[
\begin{array}{rccccccccccccc}
  p &\quad ::=\quad & a & | & b & | & x & | & (p + p) & | & (p \cdot p) & | & (p^e) \\
  v &\quad ::=\quad & u & | & x \\
  s &\quad ::=\quad & f(v,\ldots,v) & | & u & | & x \\
  e &\quad ::=\quad & s = s \\
  c &\quad ::=\quad & p = 0 & | & p < 0 & | & p \leq 0 & | & p > 0 & | & p \geq 0 & | & p \neq 0 \\
 \varphi &\quad ::=\quad & c & | & (\neg \varphi) & | &
 (\varphi\land\varphi) & | &
 (\varphi\lor\varphi) & | & 
 (\varphi\rightarrow\varphi) & | \\ &&
 (\varphi\leftrightarrow\varphi) & | &
 (\varphi\oplus\varphi)
\end{array}
\]

where $a$ is a rational number, $e$ is a natural number greater one, $b$ is a \emph{Boolean variable} and the \emph{arithmetic variable} $x$ is an inherently existential quantified and either real- or integer-valued. We call $p$ a \emph{polynomial} and use a \carl multivariate polynomial with \cln rationals as coefficients to represent it. The \emph{uninterpreted function} $f$ is of a certain \emph{order} $o(f)$ and each of its $o(f)$ arguments are either an arithmetic variable or an \emph{uninterpreted variable} $u$, which is also inherently existential quantified, but has no domain specified. Than an \emph{uninterpreted equation} $e$ has either an uninterpreted function, an uninterpreted variable or an arithmetic variable as left-hand respectively right-hand side. A \emph{constraint} $c$ compares a polynomial to zero, using a \emph{relation symbol}. Furthermore, we keep constraints in a normalized representation to be able to differ them better.

\section{Normalized constraints}
A normalized constraint has the form
\[a_1\overbrace{x_{1,1}^{e_{1,1}}\cdot\ldots\cdot x_{1,k_1}^{e_{1,k_1}}}^{m_1}+\ldots+a_n\overbrace{x_{n,1}^{e_{n,1}}\cdot\ldots\cdot x_{n,k_n}^{e_{n,k_n}}}^{m_n}\ + \ d\ \sim \ 0\]
with $n\geq0$, the \emph{$i$th coefficient} $a_i$ being an integral number ($\neq 0$), $d$ being a integral number, $x_{i,j_i}$ being a real- or integer-valued variable and $e_{i,j_i}$ being a natural number greater zero (for all $1\leq i\leq n$ and $1\leq j_i\leq k_i$). Furthermore, it holds that
$x_{i,j_i}\neq x_{i,l_i}$ if $j_i\neq l_i$ (for all $1\leq i\leq n$ and $1\leq j_i, l_i\leq k_i$) and $m_{i_1}\neq m_{i_2}$ if $i_1\neq i_2$ (for all $1\leq i_1,i_2\leq n$). If $n$ is $0$ then $d$ is $0$ and $\sim$ is either $=$ or $<$. In the former case we have the normalized representation of any variable-free consistent constraint, which semantically equals \true, and in the latter case we have the normalized representation of any variable-free inconsistent constraint, which semantically equals \false. Note that the monomials and the variables in them are ordered according the \polynomialOrder of \carl.
Moreover, the first coefficient of a normalized constraint (with respect to this order) is always positive and the greatest common divisor of $a_1,\ldots,\ a_n,\ d$ is $1$. If all variable are integer valued the constraint is further simplified to
\[\frac{a_1}{g}\cdot m_1\ +\ \ldots\ +\ \frac{a_n}{g}\cdot m_n\ + \ d'\  \sim' \ 0,\]
where $g$ is the greatest common divisor of $a_1,\ldots,\ a_n$, 
\[\sim'=\left\{
\begin{array}{ll}
\leq, &\text{ if }\sim\text{ is }< \\
\geq, &\text{ if }\sim\text{ is }> \\
\sim, &\text{ otherwise }
\end{array}
\right.\]
and
\[
d' = \left\{
\begin{array}{ll}
\lceil\frac{d}{g}\rceil &\text{ if }\sim'\text{ is }\leq \\[1.5ex]
\lfloor\frac{d}{g}\rfloor &\text{ if }\sim'\text{ is }\geq \\[1.5ex]
\frac{d}{g} &\text{ otherwise }
\end{array}
\right.\]
If additionally $\frac{d}{g}$ is not integral and $\sim'$ is $=$, the constraint is simplified $0<0$, or if $\sim'$ is $\neq$,
the constraint is simplified $0=0$.

We do some further simplifactions, such as the elimination of multiple roots of the left-hand sides in equations and inequalities with the relation symbol $\neq$, e.g., $x^3=0$ is simplified to $x=0$. We also simplify constraints whose left-hand sides are obviously positive (semi)/negative (semi) definite, e.g., $x^2\leq 0$ is simplified to $x^2=0$, which again can be simplified to $x=0$ according to the first simplification rule.

\section{Boolean combinations of constraints and Boolean variables}
A formula is stored as a directed acyclic graph, where the intermediate nodes represent the Boolean operations on the sub-formulas represented by the successors of this node. The leaves (nodes without successor) contain either a Boolean variable, a constraint or an uninterpreted equality. Equal formulas, that is formulas being leaves and containing the same element or formulas representing the same operation on the same sub-formulas, are stored only once.

The construction of formulas, which are represented by the \formulaClass, is mainly based on the presented abstract grammar. A formula being a leaf wraps the corresponding objects representing a Boolean variable, a constraint or an uninterpreted equality. A Boolean combination of Boolean variables, constraints and uninterpreted equalities consists of a Boolean operator and the sub-formulas it interconnects. For this purpose we either firstly create a set of formulas containing all sub-formulas and then construct the Formula or (if the formula shall not have more than three sub-formulas) construct the formula directly passing the operator and sub-formulas. Formulas, constraints and uninterpreted equalities are non-mutable, once they are constructed. %TODO: explain mutable member of formulas for information storage

We give a small example constructing the formula \[(\neg b\ \land\ x^2-y<0\ \land\ 4x+y-8y^7=0 )\ \rightarrow\ (\neg(x^2-y<0)\ \lor\ b ),\] with the Boolean variable $b$ and the real-valued variables $x$ and $y$, for demonstration. Furthermore, we construct the UF formula
\[v = f(u,u)\ \oplus\ w \neq u\]
with $u$, $v$ and $w$ being uninterpreted variables of not specified domains $S$ and $T$, respectively,
and $f$ is an uninterpreted function with not specified domain $T^{S\times S}$.

Firstly, we show how to create real valued (integer valued analogously with \texttt{VT\_INT}), Boolean and uninterpreted variables:
\scriptsize
\begin{verbatim}
carl::Variable x = smtrat::newVariable( "x", carl::VariableType::VT_REAL );
carl::Variable y = smtrat::newVariable( "y", carl::VariableType::VT_REAL );
carl::Variable b = smtrat::newVariable( "b", carl::VariableType::VT_BOOL );
carl::Variable u = smtrat::newVariable( "u", carl::VariableType::VT_UNINTERPRETED );
carl::Variable v = smtrat::newVariable( "v", carl::VariableType::VT_UNINTERPRETED );
carl::Variable w = smtrat::newVariable( "w", carl::VariableType::VT_UNINTERPRETED );
\end{verbatim}
\normalsize
Uninterpreted variables, functions and function instances combined in equations or inequalities comparing them are constructed the following way.
\scriptsize
\begin{verbatim}
carl::Sort sortS = smtrat::newSort( "S" );
carl::Sort sortT = smtrat::newSort( "T" );
carl::UVariable uu( u, sortS );
carl::UVariable uv( v, sortT );
carl::UVariable uw( w, sortS );
carl::UninterpretedFunction f = smtrat::newUF( "f", sortS, sortS, sortT );
carl::UFInstance f1 = smtrat::newUFInstance( f, uu, uw );
carl::UEquality ueqA( uv, f1, false );
carl::UEquality ueqB( uw, uu, true );
\end{verbatim}
\normalsize
Next we see an example how to create polynomials, which form the left-hand sides of the constraints:
\scriptsize
\begin{verbatim}
smtrat::Poly px( x );
smtrat::Poly py( y );
smtrat::Poly lhsA = px.pow(2) - py;
smtrat::Poly lhsB = smtrat::Rational(4) * px + py - smtrat::Rational(8) * py.pow(7);
\end{verbatim}
\normalsize
Constraints can then be constructed as follows:
\scriptsize
\begin{verbatim}
smtrat::ConstraintT constraintA( lhsA, carl::Relation::LESS );
smtrat::ConstraintT constraintB( lhsB, carl::Relation::EQ );
\end{verbatim}
\normalsize
Now, we can construct the atoms of the Boolean formula
\scriptsize
\begin{verbatim}
smtrat::FormulaT atomA( constraintA );
smtrat::FormulaT atomB( constraintB );
smtrat::FormulaT atomC( b );
smtrat::FormulaT atomD( ueqA );
smtrat::FormulaT atomE( ueqB );
\end{verbatim}
\normalsize
and the formulas itself (either with a set of arguments or directly):
\scriptsize
\begin{verbatim}
smtrat::FormulasT subformulasA;
subformulasA.insert( smtrat::FormulaT( carl::FormulaType::NOT, atomC ) );
subformulasA.insert( atomA );
subformulasA.insert( atomB );
smtrat::FormulaT phiA( carl::FormulaType::AND, std::move(subformulasA) );
smtrat::FormulaT phiB( carl::FormulaType::NOT, atomA )
smtrat::FormulaT phiC( carl::FormulaType::OR, phiB, atomC );
smtrat::FormulaT phiD( carl::FormulaType::IMPLIES, phiA, phiC );
smtrat::FormulaT phiE( carl::FormulaType::XOR, atomD, atomE );
\end{verbatim}
\normalsize
Note, that $\land$ and $\lor$ are $n$-ary constructors, $\neg$ is a unary constructor and all the other Boolean operators are binary.

