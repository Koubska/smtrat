\chapter{Installation}
\label{chapter:installation}
\section{Requirements}
\smtrat has been successfully compiled and tested under Linux. For its configuration we use \cmake, which can be found on \cmakeURL, and for its compilation we use \gplusplus, available on \gplusplusURL. The arithmetic calculations within \smtrat use the C++ library \ginac, which can be found on \ginacURL. The algebraic operations and data structures within \smtrat are based on the C++ library \ginacra available at \ginacraURL. Summarizing, you need the following packages before installation:
\begin{itemize}
	\item \gplusplus
	\item \cmake
	\item \ginac
	\item \ginacra
\end{itemize}

\section{Building, installing and uninstalling \smtrat}
You can download \smtrat from \smtratURL and install it the following way:

\begin{enumerate}
	\item Unpack the package:
		\begin{verbatim} tar xvzf smtrat-*.tar.gz \end{verbatim}
	\item Create a directory (e.g. build) that will contain the object files and the executables, and change into it:
		\begin{verbatim} mkdir build && cd build\end{verbatim}
	\item Configure:
		\begin{verbatim} cmake .. \end{verbatim}
	\item Build: 
		\begin{verbatim} make \end{verbatim}
	\item Install: 
		\begin{verbatim} make install \end{verbatim}
	\item Uninstall: 
		\begin{verbatim} xargs rm < install_manifest.txt \end{verbatim}
		\begin{verbatim} make clean \end{verbatim}
\end{enumerate}

More information can be found in the README file, which can be found in \smtrat directory.

%\section{Troubleshooting}
%Here we list problems, which might occur during the installation:
%\begin{enumerate}
%	\item
%\end{enumerate}
